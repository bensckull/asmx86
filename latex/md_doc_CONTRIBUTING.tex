Ce guide présentera les différentes règles de normalisation de développement qui seront utilisées dans l\textquotesingle{}optique de la réalisation de ce projet.

L\textquotesingle{}ensemble du code devra être réalisé en anglais (variables, fonctions, etc). La documentation quand à elle pourra être écrite en français.

\subsection*{Nom de classes}

Les classes utiliseront la syntaxe {\bfseries Camel\+Case}, qui implique l\textquotesingle{}utilisation de majuscules et l\textquotesingle{}absence d\textquotesingle{}espaces.


\begin{DoxyCode}
1 class File \{\}
2 
3 class ASMParser \{\}
\end{DoxyCode}


\subsection*{Nom de variables}

Les variables utiliseront la syntaxe {\bfseries underscore\+\_\+case}, qui implique l\textquotesingle{}utilisation d\textquotesingle{}underscore et d\textquotesingle{}éviter les casses majuscules.


\begin{DoxyCode}
1 const std::String full\_path;
2 std::String base\_name;
\end{DoxyCode}


\subsection*{Taille maximale des lignes}

Une ligne de code ne devra pas dépasser les {\bfseries 100 caractères}.

\subsection*{Indentations}

L\textquotesingle{}indentation se fera à l\textquotesingle{}aide de {\bfseries 4 espaces}.

\subsection*{Accolades}

Les classes et fonctions utiliseront {\bfseries un saut de ligne} avant la première accolade afin de permettre une meilleure lisibilité.

Les conditions (if, while, etc) utiliseront {\bfseries un espace} avant la première accolade.


\begin{DoxyCode}
1 class File
2 \{
3     public:
4         file(const std::String path)
5         \{
6             if(path.size() == 0) \{
7                 std::cerr << "Houston, there is a problem" << std::endl;
8             \}
9         \}
10 \}
\end{DoxyCode}


\subsection*{Documentation}

La documentation des classes et fonctions sera réalisée avec l\textquotesingle{}aide de \href{http://www.stack.nl/~dimitri/doxygen}{\tt doxygen}.

Un exemple est disponible \href{http://www.stack.nl/~dimitri/doxygen/manual/docblocks.html}{\tt à cette adresse}. 